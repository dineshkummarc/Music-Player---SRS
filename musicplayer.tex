\documentstyle[11pt]{article}

%\documentclass{article}

%\usepackage{graphicx,amscd,amsmath,amssymb,verbatim}
%\usepackage[dvips]{hyperref}
\title{Music Player and daemon}
\author{Vikash Agrawal\\Prashant Jain}
\date{\today}
\cleardoublepage
\begin{document}
\vspace{50 mm}
\pagestyle{empty}
\maketitle
\pagebreak
\tableofcontents
\pagebreak
% Document begins here
\section{Introduction}


\subsection{Purpose}
The purpose of this document is to describe all the requirements of the Music Player and daemon (MPD).
The intended audience include all developers, hobbyists and students willing to learn some new.
It will be licenced under GPLv3 and the project will be soon hosted at www.github.com.

Developers should consult this and its revisions as the only source or requirements for the project.
They should not consider any requirements statements, written or verbal as valid until they appear in this document or its revision.

\subsection{Scope}
The proposed software product is Music Player and daemon. This software would be a full fledged
music player encorporating all basic such as features to Play, Pause and Stop, or be it volume control etc. 
Moreover the selling feature of the musicplayer is to retrieve lyrics of current playing song also
broadcasting it over http and binding it to a particular port number.

The intentions of this software is to make a music player, and on the way learn, Qt using C++
, phonon backend and how to take a dip into Open Source development and contributions.

\subsection{Definitions, Acronyms, and Abbreviation}
    \begin{tabular}{ll}
        MPD & Music Player And daemon \\
        Qt  & Qt Framewors \\ 
        CP & Currently Playing Song \\
        LY & Lyrics Of the Song \\
        Ph & Phonon backend \\
        GUI & Graphical User Interface \\
				SRS & Software Requirement System\\
				Linux & Linux Operating System \\
    \end{tabular}

\subsection{Refrences}
Qt documentation, man pages and phonon library articles have been used as a refrence for this document
\subsection{Overview}
This Software Requirement Specification (SRS) is the requirements work product that formally
describes the Music Player and daemon (MPD). It includes the result of analysis done
for the project  Various techniques were used to elicit the requirements and we have identified your 
needs, analyzed and refined them. The objective of this document therefore is to 
formally describe the system’s high level requirements including 
functional requirements, non-functional requirements and business rules and constraints.
The detail structure of the documents is organized as follows:

Section 2 of the document includes description of the product, user characteristics, general constraints
and aassumptions for the Music Player. This model demonstrates the development's team understanding for the product and 
aims to mazimize the teams ability to build a system that does support the business

Section 3 presents the detail requirements which comprise the domain model.
\pagebreak
\section{General Description}
\subsection{Product Perspective}
The Music Player and daemon is a player that is fully functional which
fetches lyrics for the currently playing song and simultaneously broadcasts music
over http to a prescribed port number. 
\subsection{Product Functions}
The system functions can be described as follows:
%\bullet{\textbf{Add Music}}
\begin{itemize}
\item{\textbf{Add Music}}
User can add music to the current playing list and by this the music library will
display all the music files selected
\item{\textbf{Playing a song}}
User can Play, Pause and Stop the music from the queue. By this user can simulataneously 
view the view the length of music, and where is the current scroll using a scrollbar
\item{\textbf{Fetching Lyrics}}
The lyrics of playing song is fetched from various third party sites and this will displayed in 
the same window but in a separate text area
\item{\textbf{Broadcasting}}
This might require root permissions of from the user and then all the current playing songs
will be sent the a port number from where other users can listen it with just having the ip and the 
specified port number
\end{itemize}
\subsection{User Characteristics}
The music player will be used by end users in both Linux and Windows (using cross-compilation)
The system will be using a Graphical User Interface (GUI) and be as user friendly as possible.
\large{End User}
They are the main users of the software, and may comprimise of personel from any domain
\subsection{General Constraints}
\begin{itemize}
\item{The system has to be delivered within 3 months }
\item{The system should with compatible with Linux and Windows}
\item{The system should fetch lyrics as fast as possbile, considering network constraints}
\item{The system should be as light weight as possible}
\item{Exisisting players dont support broadcasting of music so it should be of primary concern}
\item{The system must be user friendly}
\end{itemize}
\subsection{Assumptions and Dependencies}
\begin{itemize}
\item{It is assumed that there is no compatiblity issue in the software and the Operating system}
\item{It is assumed that there is no problem in the network connectivity for the fetching of lyrics}
\item{It is assumed that the backend of the music player is well integrated and supported by the system}
\item{It is also assumed that, the output by crosses compilation runs natively in Windows}
\end{itemize}
\pagebreak

\section{Specific Requirements}
This section describes the specfic requiements of the software
\subsection{Functional Requirements}
\begin{tabular}{ll|l}
	SRS001 & Add Song & To Add music to the list \\
	SRS002 & Play & To Play the currently selected song \\
	SRS003 & Pause & To pause the currently playing song \\
	SRS004 & Stop & It will stop the currently playing song \\
	SRS005 & Lyrics & Fetch lyrics on the current \\
	~ & ~ & playing songs \\
	SRS006 & Shuffle & Playing song in shuffle mode \\
	SRS007 & Broadcasting  & Broadcast the current \\
		~ & Music & playing song over http \\
	SRS008 & ID3 tags & The music library will display all the \\
	~ & ~ & ID3 tags and for all songs \\
\end{tabular}
\subsection{Design Constraints}
\begin{tabular}{ll|l}
	SRS009 & Desktop & The system shall be a \\
	~ & Applications & desktop application \\
	SRS010 & Operating & The development will be  \\
	~ & System & done on Linux \\
	SRS011 & GUI Toolkits & Qt using C++ will be used for the \\
	~ & ~ & Graphical User Interface of the system \\
	SRS012 & Backend & The music player will use phonon \\
	~ & ~ & as the backed \\
	SRS013 & Network  & The system will use network \\
	~ & Connectivity & connectivity for fetching lyrics \\

\end{tabular}
\subsection{Non-Functional Requirements}
\subsubsection{Security}
\begin{tabular}{ll|l}
	SRS014 & Root - Login. & The system will require root \\
	~ & ~ & permissions for starting the port and cofiguring \\
	~ & ~ & it to broadcast music over that
\end{tabular}
\subsubsection{Performance}
\begin{tabular}{ll|l}
	SRS015 & Response Time & The response time to play \\
	~ & ~ &  a song should be less that 0.5 sec \\
	SRS016 & Capacity & The capacity of songs that \\
	~ & ~ & can be added to the music library shouldnot\\
	~ & ~ & restricted to any number\\
	SRS017 & Licence & The system will be licenced under GPLv3 \\
	SRS018 & User-Interface & The interface should respond\\
	~ & ~ & within 5 seconds
\end{tabular}
\subsubsection{Mainatainablity}
\begin{tabular}{ll|l}
	SRS019 & Errors & The system should keep logs of\\
	~ & ~ & all errors and crashes\\
	SRS020 & Verification & The system should include \\
	~ & ~ & test cases to test the installed system and \\
	~ & ~ & cofigurations or ports
\end{tabular}

\subsubsection{Reliablity}
\begin{tabular}{ll|l}
	SRS021 & Availablity & The system should be available\\
	~ & ~ & at all times
\end{tabular}
\end{document}
